\subsection*{当前特性}


\begin{DoxyItemize}
\item http/https server
\item http/https client
\item smtp client
\item cookie
\end{DoxyItemize}

\subsection*{依赖}


\begin{DoxyItemize}
\item boost
\item C++1y
\item libssl(https需要)
\end{DoxyItemize}

\subsection*{编译-\/安装}


\begin{DoxyItemize}
\item cd ahttpd
\item mkdir build
\item cd build
\item cmake ..
\item make
\item sudo make install
\end{DoxyItemize}

\subsubsection*{注意\+:}

若在运行时出现如下错误\+: error while loading shared libraries\+: libahttpd.\+so\+: cannot open shared object file\+: No such file or directory 请确保libahttpd.\+so所在目录在ld的搜索路径中,对于linux系统, libahttpd.\+so所在的目录一般为/usr/local/lib 若发现改目录不在ld的搜索路径中,以下步骤可让libahttpd.\+so被ld找到\+:
\begin{DoxyItemize}
\item 修改/etc/ld.so.\+conf, 加入一行/usr/local/lib
\item 执行sudo ldconfig更新配置
\end{DoxyItemize}

\subsubsection*{gcc无法编译:}


\begin{DoxyItemize}
\item 即使在最新版的gcc下编译也会出现internal compiler error, 目前已知的能编译的只有clang
\item 暂时没有workaround方案
\end{DoxyItemize}

\#\#示例 
\begin{DoxyCode}
1 \{c++\}
2 #include <ahttpd.hh>
3 struct TestHandler : public RequestHandler \{
4     void handleRequest(RequestPtr req, ResponsePtr rep) override \{
5         rep->out() << "hello world!" << std::endl;
6     \}
7 \};
8 
9 int
10 main(int argc, char *argv[])
11 \{
12     std::stringstream config("\{\(\backslash\)"http port\(\backslash\)":\(\backslash\)"8888\(\backslash\)"\}");
13     Server server(config);
14     server.addHandler("/", new TestHandler());
15     server.run();
16 \}
\end{DoxyCode}
 \subsection*{example目录中的示例\+:}

\begin{TabularC}{4}
\hline
\rowcolor{lightgray}{\bf 名称 }&{\bf 描述 }&{\bf 访问地址 }&{\bf 备注  }\\\cline{1-4}
Hello\+World &向客户端发送hello world &\href{http://127.0.0.1:8888/HelloWorld}{\tt http\+://127.\+0.\+0.\+1\+:8888/\+Hello\+World} &\\\cline{1-4}
echo &显示客户端请求包的详细信息 &\href{http://127.0.0.1:8888/echo}{\tt http\+://127.\+0.\+0.\+1\+:8888/echo} &\\\cline{1-4}
Https\+Test &https的示例 &\href{https://127.0.0.1:9999/HttpsTest}{\tt https\+://127.\+0.\+0.\+1\+:9999/\+Https\+Test} &需要输入创建密钥时的密码 \\\cline{1-4}
client &异步客户端示例 &&\\\cline{1-4}
Web\+Server &一个支持php的webserver &\href{http://127.0.0.1:8888/}{\tt http\+://127.\+0.\+0.\+1\+:8888/}\mbox{[}xxx\mbox{]} &\\\cline{1-4}
mail &使用smtp发送邮件 &&\\\cline{1-4}
proxy &http/https代理服务器 &\href{http://127.0.0.1:8888}{\tt http\+://127.\+0.\+0.\+1\+:8888} &\\\cline{1-4}
\end{TabularC}
http代理服务器地址\+: 115.\+28.\+32.\+115\+:1994

\subsection*{人丑bug多,欢迎找茬}